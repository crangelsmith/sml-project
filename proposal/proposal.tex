\documentclass{scrartcl}

\usepackage{siunitx}
\usepackage{graphicx}
\usepackage{caption}
\usepackage{subcaption}
\usepackage{glossaries}
\usepackage[english]{babel}
\usepackage{booktabs}
\usepackage[linktoc=all,hidelinks]{hyperref}
\usepackage{cleveref}
\usepackage{authblk}

% \makeatletter
% \renewcommand\AB@affilsepx{,~ \protect\Affilfont}
% \makeatother

\newcommand{\nump}[2]{\num[round-mode=places,round-precision=#2]{#1}}
\DeclareGraphicsExtensions{.pdf,.eps}
\bibliographystyle{unsrt}

\title{Proposal of Project}
\subtitle{A study of machine learning algorithms for reconstruction of missing mass in particle physics experiments.}
% \subject{Statistical Machine Learning}

\author[1]{Max Isacsson}
\author[2]{Mikael M\aa rtensson}
\author[3]{Camila Rangel Smith}
\author[4]{Henrik \"{O}hman}
\affil[1]{\small\url{max.isacsson@physics.uu.se}}
\affil[2]{\url{mikael.martensson@physics.uu.se}}
\affil[3]{\url{camila.rangel@physics.uu.se}}
\affil[4]{\url{ohman@cern.ch}}

\newacronym{ANN}{ANN}{Artificial Neural Network}

\newcommand{\etmiss}{$E_\mathrm{T}^\text{miss}$}
\newcommand{\exmiss}{$E_x^\text{miss}$}
\newcommand{\eymiss}{$E_y^\text{miss}$}

\begin{document}
\maketitle

\section{Problem statement}


\section{Background}
\subsection{Particle physics experiments}

\subsection{The Charged Higgs boson}
% Production and decay
\subsection{Missing energy}
The neutrinos in the charged Higgs boson decays only interact with matter via the weak interaction. Since the this interaction is (as its name suggests) weak, neutrinos will travel through the detector without affecting the matter that constitues it. The neutrinos can therefore only be detected in form of the absence of interaction. This absence is called missing energy.

Furthermore, since the particle beam consists of composite objects (protons) as apart from point-like objects (e.g. electrons), it is not possible to know how large portion of the beam particles' momentum go into the production of the charged Higgs boson. Therefore we can only make use of the energy and momentum conservation constraints in the directions transverse to the beam (x and y). For detected particles, the momentum can be reconstructed in all three spatial directions, while for the neutrinos this information is completely unknown. In the end we are left with two quantities: \exmiss\ and \eymiss\.

\section{The Dataset}
\subsection{Structure}

\subsection{Production}


\section{Solution strategy}
We will perform a 
% Regression
% Artificial neural network, bayesian regression, support vector machine, gaussian process.


%\newpage
% \bibliography{ref}

\end{document}
